\documentclass{article}
\usepackage{amsmath}
\usepackage{amsthm}
\usepackage{amssymb}
\usepackage[margin=0.5in]{geometry}
\usepackage[parfill]{parskip}

\title{Problem 5: Bayes Math}
\author{}
\date{}

\begin{document}

\maketitle

\section*{A. Prove that:}

\begin{equation}
P(A|B,C) = \frac{P(B|A,C) \cdot P(A|C)}{P(B|C)}
\end{equation}

As the definition of conditional probability:
\begin{equation}
P(A|B,C) = \frac{P(A \cap B \cap C)}{P(B \cap C)}
\end{equation}

By applying the chain rule, we can express the numerator as:
\begin{equation}
P(A \cap B \cap C) = P(B|A,C) \cdot P(A \cap C)
\end{equation}

It is also useful to express $P(A \cap C)$ using the definition of conditional probability:
\begin{equation}
P(A \cap C) = P(A|C) \cdot P(C)
\end{equation}

Therefore:
\begin{equation}
P(A \cap B \cap C) = P(B|A,C) \cdot P(A|C) \cdot P(C)
\end{equation}

Similarly,:
\begin{equation}
P(B \cap C) = P(B|C) \cdot P(C)
\end{equation}

Taking back into the original expression:
\begin{equation}
P(A|B,C) = \frac{P(B|A,C) \cdot P(A|C) \cdot P(C)}{P(B|C) \cdot P(C)}
\end{equation}

Thus, we can cancel $P(C)$ from the numerator and denominator:
\begin{equation}
P(A|B,C) = \frac{P(B|A,C) \cdot P(A|C)}{P(B|C)}
\end{equation}

\section*{B.}
\subsection*{Setup:}
\begin{itemize}
    \item $A$ denote the event that the coin is fair.
    \item $B$ denote the event that the coin is double-headed.
    \item Prior probabilities: $P(A) = \frac{F}{F+1}$ and $P(B) = \frac{1}{F+1}$.
    \item For fair coin, $P(H|A) = \frac{1}{2}$.
    \item For double-headed coin, $P(H|B) = 1$.
\end{itemize}

\subsection*{Applying Bayes' Theorem:}
We want to find $P(B|n\text{ heads})$, after observing $n$ heads.

As per Bayes' theorem:
\begin{equation}
P(B|n\text{ heads}) = \frac{P(n\text{ heads}|B) \cdot P(B)}{P(n\text{ heads})}
\end{equation}

Compute the likelihoods:
\begin{align}
P(n\text{ heads}|B) &= 1^n = 1 \\
P(n\text{ heads}|A) &= \left(\frac{1}{2}\right)^n
\end{align}

Using the law of total probability:
\begin{align}
P(n\text{ heads}) &= P(n\text{ heads}|A) \cdot P(A) + P(n\text{ heads}|B) \cdot P(B) \\
&= \left(\frac{1}{2}\right)^n \cdot \frac{F}{F+1} + 1 \cdot \frac{1}{F+1} \\
&= \frac{1}{F+1}\left[\frac{F}{2^n} + 1\right]
\end{align}

Thus, 
\begin{align}
P(B|n\text{ heads}) &= \frac{1 \cdot \frac{1}{F+1}}{\frac{1}{F+1}\left[\frac{F}{2^n} + 1\right]} \\
&= \frac{1}{\frac{F}{2^n} + 1}
\end{align}

\subsection*{Find the smallest $n$ such that $P(B|n\text{ heads}) > 0.5$:}
We need to solve:
\begin{equation}
\frac{1}{\frac{F}{2^n} + 1} > 0.5
\end{equation}
Solving this inequality:
\begin{equation}
    n > \log_2(F)
\end{equation}

Since $n$ must be an integer, we take the ceiling:
\begin{equation}
n \geq \lceil \log_2(F) \rceil
\end{equation}

\end{document}