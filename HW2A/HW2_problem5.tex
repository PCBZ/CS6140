\documentclass[12pt]{article}
\usepackage{amsmath}
\usepackage[margin=0.7in]{geometry}
\usepackage{parskip}

\title{Problem 5}
\author{}
\date{}

\begin{document}
\maketitle
\thispagestyle{empty}

Considering a binary classification task where each instance is assigned a score by the classifier.

\subsection*{Notation:}
\begin{itemize}
  \item Let $P = \{x_1^+, x_2^+, \ldots, x_m^+\}$ be the set of $m$ positive instances.
  \item Let $N = \{x_1^-, x_2^-, \ldots, x_n^-\}$ be the set of $n$ negative instances.
  \item Let $s(x)$ be the score assigned to instance $x$ by the classifier.
\end{itemize}

The AUC is calculated as the area under the curve defined by plotting the True Positive 
Rate (TPR) against the False Positive Rate (FPR) for all possible classification thresholds.

\subsection*{Alternative Definition of AUC}

Defining a function $\phi(x^+, x^-)$ over positive and negative instance pairs:
\[
\phi(x^+, x^-) =
\begin{cases}
1 & \text{if } s(x^+) > s(x^-) \\
0.5 & \text{if } s(x^+) = s(x^-) \\
0 & \text{if } s(x^+) < s(x^-)
\end{cases}
\]

Then the AUC can be defined as:

\[
\text{AUC} = \frac{1}{mn} \sum_{x^+ \in P} \sum_{x^- \in N} \phi(x^+, x^-)
\]

This definition calculates the proportion of positive-negative pairs $(x^+, x^-)$ such that the positive instance has a higher score than the negative instance.

\subsection*{Equivalence to ROC Area}
To see why this is equivalent to the ROC area:

\begin{itemize}
  \item The ROC curve is constructed by varying the classification threshold from $+\infty$ to $-\infty$, and at each point computing TPR and FPR.
  \item Each time a positive instance is passed over, TPR increases by $\frac{1}{m}$.
  \item Each time a negative instance is passed over, FPR increases by $\frac{1}{n}$.
\end{itemize}

Thus, the ROC curve is a step function consisting of $m$ vertical steps and $n$ horizontal steps. The area under the ROC curve corresponds to the expected TPR for each FPR step, which can be interpreted as:
\[
\text{AUC} = \Pr(s(x^+) > s(x^-)) + 0.5 \cdot \Pr(s(x^+) = s(x^-))
\]

This is exactly the average value of $\phi(x^+, x^-)$ over all $mn$ positive-negative pairs, as defined above.
\subsection*{Conclusion}

Therefore, the AUC is equal to the proportion of positive-negative pairs that are correctly ranked by the classifier. This provides both a probabilistic and geometric interpretation of the AUC metric.


\end{document}